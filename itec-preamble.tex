% %%%%%%%%%%%%%%%%%%%%%%%%%%%%%%%%%%%%%%%%%%%%%%%%%%%%%%%%%%%%%%%%%%%%%%
%  ╭─────────╮                      ╔══╦╗ ╗ ╔═╦═╗ ╥ ┌─┐┌─┐┌─┐┬ ┬┌─┐┌┐ ┬
%  │ ,-= ━━━┑│                  |   ╠═╦╝╚╗╠╗║ ║ ╠═╣ ├─┤├─┤│  ├─┤├─ │└┐│
%  │ % iTec  │                  |   ╨ ╚═ ╚╝╚╝ ╨ ╨ ╨ ┴ ┴┴ ┴└─┘┴ ┴└─┘┴ └┘
%  │┃°. .°.  │ Chair Individual |          ┬ ┬┌┐ ┬┬┬  ┬┌─┐┬─┐┌─┐┬┌┬┐┬ ┬
%  │┖  °   ° │   and Technology |          │ ││└┐││└┐┌┘├─ ├┬┘└─┐│ │ └┬┘
%  ╰─────────╯                             └─┘┴ └┘┴ └┘ └─┘┴└─└─┘┴ ┴  ┴
% %%%%%%%%%%%%%%%%%%%%%%%%%%%%%%%%%%%%%%%%%%%%%%%%%%%%%%%%%%%%%%%%%%%%%%
%  This file is part of the Master's thesis LaTeX template used at the
%  Chair Individual and Technology (iTec) at RWTH Aachen University.
% %%%%%%%%%%%%%%%%%%%%%%%%%%%%%%%%%%%%%%%%%%%%%%%%%%%%%%%%%%%%%%%%%%%%%%

%% %%%%%%%%%%%%%%%%%%%%%%%%%%%%%%%%%%%%%%%
%%  PACKAGES & SETTINGS
%% %%%%%%%%%%%%%%%%%%%%%%%%%%%%%%%%%%%%%%%

%% encoding
\usepackage[utf8]{inputenc}
\usepackage[T1]{fontenc}

%% font(s)
\usepackage{microtype}

%% change standard fonts to Palatino and Helvetica
\usepackage{palatino}

%% symbols
\usepackage{latexsym}
\usepackage{amsmath}
\usepackage{amssymb}

%% table of contents
%\usepackage[tocflat]{tocstyle}
%\usepackage{tocstyle}
%\usetocstyle{standard}
%\setcounter{tocdepth}{3}

%% tables
\usepackage{booktabs}

%% color & graphics
\usepackage{color}
\usepackage{graphicx}
\graphicspath{{imgs/},{pics/},{tikz/}}

%% spacing
%% do not indent at new paragraphs but add a vertical offset
%\usepackage{noindent}
\setlength{\parindent}{0mm}
\addtolength{\parskip}{\baselineskip}


%% bibliography
\usepackage{natbib}

%Zitierbefehle
%citation commands
\newcommand{\fullcite}{\citep} %for "Author [1980]"
\renewcommand{\citeyear}{\citeyearpar} %for "[1980]"

%% package for control structures
\usepackage{ifthen}

%% marginpar hack --- moves margin notes to correct position
\usepackage{mparhack}

\usepackage{lipsum}
\usepackage{todonotes}

%% URLs/hyperref
%%make readable references
\usepackage[pdftex,plainpages=false,pdfpagelabels]{hyperref}

\usepackage{comment}


%% %%%%%%%%%%%%%%%%%%%%%%%%%%%%%%%%%%%%%%%
%%  header / footer
%% %%%%%%%%%%%%%%%%%%%%%%%%%%%%%%%%%%%%%%%
%---------------------<Layout in the style of "A Pattern Approach to Interaction Design>---------------------------

% Change page headers and footers:
\usepackage{fancyhdr}
\pagestyle{fancy}
\fancyhf{}
\fancyhead[RE]{\slshape \nouppercase{\leftmark}}    % Even page header: "page   chapter"
\fancyhead[LO]{\slshape \nouppercase{\rightmark}}   % Odd  page header: "section   page"
\fancyhead[RO,LE]{\bfseries \thepage} 
\renewcommand{\headrulewidth}{1pt}    % Underline headers
\renewcommand{\footrulewidth}{0pt}    

\fancypagestyle{plain}{               % No chapter+section on chapter start pages
\fancyhf{}
\fancyhead[RO,LE]{\bfseries \thepage}
\renewcommand{\headrulewidth}{1pt}
\renewcommand{\footrulewidth}{0pt}
}

% Left headings: "1  INTRODUCTION"
\renewcommand{\chaptermark}[1]{%
\markboth{\thechapter\ \ \ \ #1}{}}

% Right headings: "1.1  Basics"
\renewcommand{\sectionmark}[1]{%
\markright{\thesection\ \ \ \ #1}{}}

% some Fancyhdr problem...
\addtolength{\headheight}{2pt} % To avoid overfull vboxes from fancyhdr


%creating a better way to change the layout for the abstract pages
\usepackage{geometry}

% \ifthenelse{\lengthtest{\paperheight=250mm}}%
% {% -----------------B5 Layout-----------------
% % Page layout

% %\pdfpageheight250mm
% %\pdfpagewidth176mm
% \geometry{ b5paper,
%              top = 27mm,
%              footskip = 10mm,
%              inner = 19mm,
%              outer = 39mm,
%              textheight = 175mm,
%              textwidth = 84mm,
%              marginparsep = 3mm,
%              marginparwidth = 32mm
% }
% \savegeometry{myText}
% % -----------------/ B5 Layout-----------------
% }%
% {% -----------------A4 Layout-----------------
% % Page layout
% \geometry{ a4paper,
%              twoside,
%              includemp,
%              includehead,
%              top = 30mm,
%              headsep = 10mm,
%              bindingoffset = 10mm,
%              inner = 20mm,
%              outer = 40mm,
%              bottom = 45mm,
%              marginparsep = 10mm,
%              marginparwidth = 30mm
% }
% \savegeometry{myText}
% % -----------------/ A4 Layout-----------------
% }
% % Abstract layout
% \geometry{marginparsep   = 0mm,
%           marginparwidth = 0mm
% }
% \savegeometry{myAbstract}
% \loadgeometry{myText}

\newlength{\fullwidth} % Width of text plus margin notes
\setlength{\fullwidth}{\textwidth}
%\addtolength{\fullwidth}{\marginparsep}
%\addtolength{\fullwidth}{\marginparwidth}
%
\setlength{\headwidth}{\fullwidth} % Header stretches over margin notes

%% %%%%%%%%%%%%%%%%%%%%%%%%%%%%%%%%%%%%%%%
%% memoir style geometry settings from here:
%% https://tex.stackexchange.com/questions/331855/set-thesis-geometry

\geometry{
  inner=37.125mm,
  outer=33.4125mm,
  top=37.125mm,
  bottom=37.125mm,
  heightrounded,
  marginparwidth=51pt,
  marginparsep=17pt,
  headsep=24pt,
}

%---------------------</Layout in the style of "A Pattern Approach to Interaction Design>---------------------------

%needed for the full-face titlepage
\usepackage{eso-pic}

%index verwenden
%make an index
\usepackage{makeidx}
\makeindex

%Index Formatierungshilfen
%formatting helpers for the index
\newcommand{\uu}[1]{\underline{#1}}
\newcommand{\ii}[1]{\textit{#1}}

%neue Definition der Index Umgebung
%redesign of the index
\renewenvironment{theindex}{%
  \vspace*{50pt}%
  {\Huge\bfseries\indexname}\par%
  \vspace*{40pt}%
  \setlength{\parskip}{0pt}%
  \setlength{\parindent}{0pt}%
  \small%
  \renewcommand{\item}{\par{}}%
  \renewcommand{\subitem}{\par\hspace{2em}- }%
}%
{}

%Maximale Gliederungstiefe, die noch ins Inhaltsverzeichnis aufgenommen wird
%maximum depth for the table of contents
\setcounter{tocdepth}{3}

%Vorschlag fuer ein schoenes Farbschema
%Set of colors which look nice together
\usepackage{color}
\definecolor{orange_light}{rgb}{1,0.8,0.4}
\definecolor{orange_med}{rgb}{0.753,0.62,0.373}
\definecolor{orange_dark}{rgb}{0.506,0.412,0.251}

\definecolor{green_light}{rgb}{0.8,1,0.4}
\definecolor{green_med}{rgb}{0.635,0.745,0.376}
\definecolor{green_dark}{rgb}{0.435,0.498,0.255}

\definecolor{blue_light}{rgb}{0.4,0.8,1}
\definecolor{blue_med}{rgb}{0.365,0.624,0.749}
\definecolor{blue_dark}{rgb}{0.251,0.42,0.502}

\definecolor{pink_light}{rgb}{1,0.435,0.812}
\definecolor{pink_med}{rgb}{0.745,0.38,0.62}
\definecolor{pink_dark}{rgb}{0.498,0.255,0.416}

\definecolor{yellow_light}{rgb}{1,1,0.4}
\definecolor{yellow_med}{rgb}{0.757,0.745,0.373}
\definecolor{yellow_dark}{rgb}{0.506,0.49,0.251}

%blue (for URLs)
\definecolor{blue}{rgb}{0,0,1}

%we need this to determine if a figure is on an odd or even page
\usepackage{chngpage}

%we need this to redesign the captions
\usepackage[font=normalsize,labelfont=bf]{caption}

%if a figure takes more than 85% of a page it will be typeset on a separate page
\renewcommand{\floatpagefraction}{0.85}

%we need this to rotate big figures
\usepackage[figuresright]{rotating}

%pre-defined matlab code formats
\usepackage{alltt}
\definecolor{string}{rgb}{0.7,0.0,0.0}
\definecolor{comment}{rgb}{0.13,0.54,0.13}
\definecolor{keyword}{rgb}{0.0,0.0,1.0}

%----------------------------------------------------------------------------------
% \myTxtRef     { LABLE }
%
%Referenz auf Kapitel oder Abschnitte - gibt nummer und namen aus, z.B.: 5.3---"Yaddahyaddah"
%references chapters or sections, outputs number and title, e.g., 5.3---"Yaddahyaddah"
\newcommand{\myTxtRef}[1]
{%
        \ref{#1} ``\nameref{#1}''%
}

%----------------------------------------------------------------------------------
% \myTxtRefPP   { LABLE }
%
%Referenz auf Kapitel oder Abschnitte - gibt nummer, namen und seiten aus, z.B.: 5.3---"Yaddahyaddah" (p. 45)
%references chapters or sections, outputs number and title, e.g., 5.3---"Yaddahyaddah"
\newcommand{\myTxtRefPP}[1]
{%
        \ref{#1} ``\nameref{#1}'' (p.~\pageref{#1})%
}


%----------------------------------------------------------------------------------
% \chapterquote { QUOTATION }
%                               { SOURCE }
%
%outputs a quote with its source, can be used as an introduction to chapters
\newcommand{\chapterquote}[2]{
\begin{quotation}
    \begin{flushright}
        \noindent\emph{``{#1}''\\[1.5ex]---{#2}}
    \end{flushright}
\end{quotation}
}

%----------------------------------------------------------------------------------
% \emptydoublepage
%
% Clear double page without any header or footer at end of chapters
\newcommand{\emptydoublepage}{\clearpage\thispagestyle{empty}\cleardoublepage}

%----------------------------------------------------------------------------------
% \pagebreak    [ SOME STRANGE LATEX VALUE ]
%
%pagebreaks for the final print version (last resort weapon against wrong pagebreaks by LaTeX)
\newcommand{\PB}[1][3]{\pagebreak[#1]}

%----------------------------------------------------------------------------------
% \TM
%Places a (TM) symbol
\newcommand{\TM}{\textsuperscript{\texttrademark}}

