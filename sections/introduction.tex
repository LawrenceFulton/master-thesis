% %%%%%%%%%%%%%%%%%%%%%%%%%%%%%%%%%%%%%%%%%%%%%%%%%%%%%%%%%%%%%%%%%%%%%%
%  ╭─────────╮                      ╔══╦╗ ╗ ╔═╦═╗ ╥ ┌─┐┌─┐┌─┐┬ ┬┌─┐┌┐ ┬
%  │ ,-= ━━━┑│                  |   ╠═╦╝╚╗╠╗║ ║ ╠═╣ ├─┤├─┤│  ├─┤├─ │└┐│
%  │ % iTec  │                  |   ╨ ╚═ ╚╝╚╝ ╨ ╨ ╨ ┴ ┴┴ ┴└─┘┴ ┴└─┘┴ └┘
%  │┃°. .°.  │ Chair Individual |          ┬ ┬┌┐ ┬┬┬  ┬┌─┐┬─┐┌─┐┬┌┬┐┬ ┬
%  │┖  °   ° │   and Technology |          │ ││└┐││└┐┌┘├─ ├┬┘└─┐│ │ └┬┘
%  ╰─────────╯                             └─┘┴ └┘┴ └┘ └─┘┴└─└─┘┴ ┴  ┴
% %%%%%%%%%%%%%%%%%%%%%%%%%%%%%%%%%%%%%%%%%%%%%%%%%%%%%%%%%%%%%%%%%%%%%%
%  This file is part of the Master's thesis LaTeX template used at the
%  Chair Individual and Technology (iTec) at RWTH Aachen University.
% %%%%%%%%%%%%%%%%%%%%%%%%%%%%%%%%%%%%%%%%%%%%%%%%%%%%%%%%%%%%%%%%%%%%%%

%% %%%%%%%%%%%%%%%%%%%%%%%%%%%%%%%%%%%%%%%%%%%%%%%%%%%%%%%%%%%%%%%%%%%%%
\chapter{Introduction}
\label{chap:introduction}

\section{Context and Motivation}

Large Language Models (LLMs) such as GPT  \citep{brown2020language} or Mistral \citep{jiang2023mistral7b} have rapidly evolved from simple text predictors into complex conversational agents capable of reasoning, debating, and simulating social interaction. Their ability to generate coherent, persuasive, and context-aware arguments has led to increasing interest in using LLMs to study communication dynamics, decision-making, and even human-like social behaviour. Within computational social science, these models offer a unique opportunity: they allow controlled, repeatable simulations of debates that are otherwise difficult to study empirically with humans.

However, this opportunity comes with risks. Despite their fluency, LLMs inherit and sometimes amplify social biases present in their training data. Research has already shown that such models can display systematic patterns of gender, racial, and ideological bias in text generation, evaluation, and interaction. When deployed in multi-agent settings, such as in simulate debates or discussions, these biases may subtly affect how arguments are presented, whose perspectives appear more dominant, or which positions seem more persuasive.

Understanding such effects is crucial not only for assessing the fairness of LLM-driven systems but also for improving their interpretability as tools in social simulation. If an LLM debate environment systematically treats gendered personas differently, it may distort results, reinforcing stereotypes rather than modelling neutral discourse. Investigating these dynamics therefore matters both for advancing methodological rigour in computational social systems and for ensuring that the use of LLMs in social contexts remains transparent, responsible, and scientifically grounded.


\section{Problem Statement and Research Gap}

Recent advances in large language models (LLMs) have made it possible to simulate complex social phenomena such as political debates through multi-agent interactions. These simulations promise a controlled and reproducible way of studying opinion formation and persuasion processes—key interests in computational social science. However, while LLMs can generate coherent dialogue, the extent to which their simulated behaviours reliably reflect the intended social constructs rather than artefacts of the underlying models remains unclear.

A recent contribution in this area is the work by \citet{taubenfeld_systematic_2024}, who investigated systematic biases in simulated dyadic and triadic debates between LLM-based agents. Their framework paired agents with opposing ideological profiles (Republican, Democrat, and Default) and analysed how their expressed opinions evolved over multiple dialogue turns when debating controversial topics (\textit{Gun Violence, Racism, Climate Change, and Illegal Immigration}). The study found that, despite being instructed to represent distinct viewpoints, agents tended to converge in their opinions. When instructed to represent the same viewpoint, however, their opinions gradually shifted. The authors hypothesised that this behaviour reflects the underlying bias of the base model and noted that this goes against the common theory of echo chambers, where agents with like-minded views would tend to intensify their beliefs when interacting with each other.

To test this hypothesis, the authors fine-tuned separate versions of the model to adopt either a Democratic or Republican perspective. This process involved (1) generating 20 Democratic and 20 Republican personas, (2) prompting these personas to answer 100 politically themed questions designed to elicit their beliefs and reasoning, and (3) fine-tuning the base model on the resulting 2,000 responses using a parameter-efficient QLoRA next-word prediction task.

With these Democratic- and Republican-tuned models, the authors repeated the debate simulations. The results showed not only continued convergence between agents but, crucially, convergence towards the ideological direction of the fine-tuned model. This outcome strengthened the hypothesis that LLM debate behaviour is shaped more by the base model’s internal bias than by the explicit persona instructions given to the agents.

While the study established an important foundation, several broader questions remain open. First, the degree to which LLM behaviour depends on prompt design is largely unexplored. Subtle differences in phrasing, politeness, or structural cues could influence how agents interpret their roles and respond to debate topics, yet most studies—including that of \citet{taubenfeld_systematic_2024}—use a single prompt structure without assessing its robustness.

Second, the dynamics of opinion change observed in debate settings require a more quantitative treatment. The original analysis relied primarily on qualitative interpretation of plotted opinion trajectories. It remains unclear whether the apparent convergence toward a shared stance represents meaningful interaction effects or merely reflects model-level bias.

Finally, the debate simulations examined by \citet{taubenfeld_systematic_2024} were restricted to a bipartisan context (Republican versus Democrat). Real-world political discourse, however, is often more complex, involving multiple parties and overlapping ideological positions. Extending such analyses to a multi-party system would provide a more comprehensive test of LLMs’ ability to sustain diverse and stable viewpoints.

In addition, the original study relied on now outdated models (\texttt{gpt-3.5-turbo-instruct}, \texttt{Mistral 7B}, and \texttt{Solar 10.7B}), raising questions about whether the observed effects persist in more recent architectures. Addressing these methodological and conceptual gaps is crucial for understanding when and how LLM-based debate simulations can serve as valid tools for social-scientific inquiry.


\section{Research Questions}
\label{sec:research_questions}
While prior work often relies on a single debate prompt, subtle linguistic variations can alter how an LLM interprets its role and responds to contentious topics. This question examines whether different prompt phrasings—while keeping the underlying task identical—lead to measurable differences in agents’ expressed opinions. Assessing prompt sensitivity is essential for determining the robustness and reproducibility of LLM-based debate simulations. This leads to the first research question:
\researchquestion{RQ1: Is there a significant impact of the prompt chosen on the opinions?}



Unlike previous studies that constructed political personas through ad-hoc prompting, this study derives personas empirically from the German Longitudinal Election Study (GLES) dataset. These data-driven personas allow for a more grounded representation of real-world political attitudes. This second reseach-question therefore tests whether LLM agents initialised with empirically informed political profiles maintain ideological consistency with their affiliated party when engaged in debate and is formulated as follows.
\researchquestion{RQ2: Do agents with strong party affiliation align with their party’s official stance?}  


Earlier findings by \citet{taubenfeld_systematic_2024} indicated that LLM agents tend to converge in opinion, even when representing opposing sides. This question revisits that phenomenon using newer models and an extended multi-party debate framework. The aim is to quantify whether opinion change over time reflects genuine conversational influence between agents or merely the manifestation of shared underlying model bias.
\researchquestion{RQ3: Does discussion cause agents' opinions to significantly shift from their starting points and converge towards more uniform stances?}  




\section{Approach Summary}

This study simulates dyadic political debates between Large Language Model (LLM) agents endowed with socio-demographic and political profiles derived from the German Longitudinal Election Study (GLES). The methodological pipeline consists of six main stages (Figure~\ref{fig:approach_overview}):

\begin{figure}[h]
\centering
\begin{tikzpicture}[
    node distance = 14mm and 10mm,
    every node/.style = {align=center, font=\small, rounded corners, minimum height=1.1cm, text width=3.2cm, draw=black!70, fill=gray!5},
    arrow/.style = {->, thick, >=latex, black!70}
]

% Top row
\node (step1) {1. Agent Profiling \\ and Sampling};
\node (step2) [below=of step1] {2. Topic \\ Selection};
\node (step3) [right=of step2] {3. Debate \\ Pairing Design};

% Bottom row
\node (step5) [right=of step3] {5. Debate Simulation \\ and Data Collection};
\node (step4) [above=of step5] {4. Prompt Construction \\ and Sensitivity};
\node (step6) [below=of step5] {6. Validation \\ and Statistical Analysis};

% Arrows (top row)
\draw[arrow] (step1) -- (step3);
\draw[arrow] (step2) -- (step3);

% Arrows (bottom row)
\draw[arrow] (step4) -- (step5);
\draw[arrow] (step5) -- (step6);

% Vertical connectors
\draw[arrow] (step3) -- (step4);

\end{tikzpicture}
\caption{Overview of the six-stage methodological pipeline. }
\label{fig:approach_overview}
\end{figure}


\begin{enumerate}
    \item \textbf{Agent Profiling and Sampling:}  
    Personas are generated using the principle of \emph{Quantitative Persona Creation} \citep{Salminen2020quantative}. Each persona represents an archetypical supporter of one of six German political parties (CDU/CSU, SPD, Bündnis 90/Die Grünen, FDP, AfD, Die Linke), based on socio-demographic and attitudinal variables from the GLES 2021 dataset. A default, non-biographical agent serves as a baseline reference.

    \item \textbf{Topic Selection:}  
    Debate topics are drawn from five politically salient statements taken from the 2021 \emph{Wahl-O-Mat}, Germany’s official voter-assistance tool. These topics were selected to cover a range of policy domains and to elicit divergent party opinions.

    \item \textbf{Debate Pairing Design:}  
    Agents are paired across all within- and between-party combinations, including debates with the default agent. Each pair discusses all five topics, leading to a comprehensive coverage of ideological interactions.

    \item \textbf{Prompt Construction and Sensitivity Conditions:}  
    To examine the robustness of model responses, three prompt formulations—varying in conciseness, descriptiveness, and politeness—are applied to two use cases: debate continuation and questionnaire evaluation. This results in six total prompt skeletons (see Table~\ref{tab:prompt_skeletons}).

    \item \textbf{Debate Simulation and Data Collection:}  
    Debates proceed for 20 alternating turns using a modified version of the SAUCE framework \citep{neuberger2024sauce}. Every fourth turn, both agents are queried with a Likert-style evaluation of the current topic to capture temporal opinion shifts. Across all conditions, this yields 21,000 individual measurements.

    \item \textbf{Validation and Statistical Analysis:}  
    To assess the validity of the generated Likert responses, a subset of outputs is evaluated by human annotators. The main quantitative analyses employ mixed-effects models to test for effects of prompt version, debate partner, and time on agents’ expressed opinions, accounting for repeated measures within debate sessions.
\end{enumerate}

This multi-stage design ensures a systematic investigation of (i) ideological alignment, (ii) opinion shift dynamics, and (iii) prompt sensitivity across different LLM architectures and interaction contexts.



\section{Main Contributions}

This thesis contributes to the emerging field of LLM-based social simulations by extending existing debate frameworks both conceptually and methodologically. Building on \citet{taubenfeld_systematic_2024}, it introduces several innovations that enhance the validity, robustness, and generalizability of such simulations.

First, the study systematically investigates prompt robustness. While prior work typically relied on a single debate prompt, this thesis varies the linguistic framing of prompts while keeping all other parameters constant. This enables an empirical assessment of how sensitive debate outcomes are to subtle differences in wording, and whether observed opinion differences stem from genuine model bias or the phrasing of the task itself.

Second, it develops an empirical method for persona construction. Instead of generating political profiles through ad-hoc instructions, the personas used in this study are derived from the German Longitudinal Election Study (GLES) dataset. This approach grounds the agents’ ideological orientations in real-world survey data, creating a reproducible and data-driven foundation for representing political attitudes.

Third, the thesis extends the traditional bipartisan debate setup to a multi-party framework that mirrors the diversity of the German political landscape. This expansion allows for a more realistic and fine-grained analysis of ideological alignment, cross-party convergence, and the stability of multiple competing viewpoints within the same discussion.

Finally, the study introduces a quantitative approach to opinion dynamics using mixed-effects models. This enables a statistical assessment of whether debates lead to significant opinion shifts and whether convergence arises from conversational influence or shared underlying model bias. In doing so, the analysis moves beyond qualitative interpretation towards a more rigorous, data-driven understanding of interaction effects.

Together, these contributions advance the methodological toolkit for studying social phenomena through language model interactions and help clarify under what conditions LLM debates can serve as valid approximations of human political discourse.



\section{Thesis outline}

