

\chapter{Types of harm}
\label{abc}
\begin{table}[h!]
    \centering
    \renewcommand{\arraystretch}{1.3}
    \begin{tabular}{|p{4cm}|p{10cm}|}
        \hline
        \multicolumn{2}{|c|}{\textbf{Representational Harms}} \\
        \hline
        \textbf{Type of Harm} & \textbf{Definition} \\
        \hline
        \textbf{Derogatory language} & Pejorative slurs, insults, or phrases that target and denigrate a social group.\\
        \hline
        \textbf{Disparate system performance} & Degraded understanding, diversity, or richness in language processing or generation between social groups.
         \\
        \hline
        \textbf{Erasure} & Omission or invisibility of the language and experiences of a social group.
        \\
        \hline
        \textbf{Exclusionary norms} & Reinforced normativity of the dominant social group and implicit exclusion or devaluation of other groups.  \\
        \hline
        \textbf{Misrepresentation} & An incomplete or non-representative distribution of the sample population generalized to a social group. \\
        \hline
        \textbf{Stereotyping} & Negative, generally immutable abstractions about a labeled social group. \\
        \hline
        \textbf{Toxicity} & Offensive language that attacks, threatens, or incites hate or violence against a social group. \\
        \hline
        \multicolumn{2}{|c|}{\textbf{Allocational Harms}} \\
        \hline
        \textbf{Type of Harm} & \textbf{Definition} \\
        \hline
        \textbf{Direct discrimination} & Disparate treatment due explicitly to membership of a social group. \\
        \hline
        \textbf{Indirect discrimination} & Disparate treatment despite formally neutral consideration towards social groups, due to proxies or other implicit factors. \\
        \hline
    \end{tabular}
    \caption{Types of Representational and Allocational Harms by \citet[p.1103]{gallegos_bias_2024}}
    \label{tab:harms}
\end{table}
